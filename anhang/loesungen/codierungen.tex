\section{\autoref{chapter-codierungen}, \nameref{chapter-codierungen}}

\subsection*{Aufgaben 1}

\begin{enumerate}
\item \autoref{table-shortest-words} zeigt alle 39 Wörter.

\begin{table}[htb]
\centering
\begin{tabular}{|l|l|l|l|l|l|l|l|}
\hline
a   & b   & c   & aa  & ab  & ba  & bb  & ac  \\ \hline
ca  & bc  & cb  & cc  & aaa & aab & aba & abb \\ \hline
baa & bab & bba & bbb & aac & aca & acc & caa \\ \hline
cac & cca & ccc & bbc & bcb & bcc & cbb & cbc \\ \hline
ccb & abc & acb & bac & cab & bca & cba &     \\ \hline
\end{tabular}
\caption{Die Wörter der Länge $1$, $2$ und $3$ müssen aufgezählt werden.}
\label{table-shortest-words}
\end{table}

\item Das Wort $abba$ ist aus $\mathscr{A}_{abc}^*$. Das Wort $abbdca$ ist nicht aus $\mathscr{A}_{abc}^*$, da $d$ nicht in $\mathscr{A}_{abc}$ vorkommt.
\item Nein, da \say{e} und \say{n} nicht in $\mathscr{A}_{Lat} = \{A, B, C, D, \dots , X, Y, Z\}$ vorkommen und somit kann Ben nicht aus $\mathscr{A}_{Lat}^*$ sein, da das Wort nicht in dieser Menge ist.
\item Das Wort $abbaab$ kommt in $\mathscr{A}_{abc}^6$ vor. Darin sind alle Wörter der Länge 6 enthalten die aus den Zeichen $a$ und $b$ bestehen. Das Wort $abbdca$ ist nicht aus $\mathscr{A}_{abc}^*$ und kann somit auch nicht aus $\mathscr{A}_{abc}^6$ sein.
\item Es sind die Wörter aus $\mathscr{A}_{abc}^2$ gesucht. Dies bedeutet alle Wörter der Länge $2$. \autoref{table-shortest-words} zeigt diese Wörter bereits in den ersten beiden Zeilen (von aa bis cc).
\end{enumerate}

\subsection*{Aufgaben 2}

\begin{enumerate}
\item INFORMATIK
\item $\cdot~-~-~\cdot/~\cdot~/~-~/~\cdot~/~\cdot~-~\cdot$ (PETER)
\end{enumerate}

\subsection*{Aufgaben 3}

\begin{enumerate}
\item Nein, der Morsecode ist kein Blockcode. Die Code-Wörter sind unterschiedlich lang.
\item Jedes Code-Wort kommt nur einmal vor. Man muss das komplette Code-Wort lesen damit korrekt decodiert wird.
\item Zwischen zwei Code-Wörtern wird eine Pause eingefügt (3. Zeichen im Morsealphabet $/$).
\item \autoref{table-arrows} zeigt den Code.
\end{enumerate}

\begin{table}[htb]
\centering
\begin{tabular}{|l|l|}
\hline
Zeichen & Code-Wort \\ \hline
  \faAngleDoubleDown      & 000       \\ \hline
  \faAngleDoubleLeft      & 001       \\ \hline
  \faAngleDoubleRight     & 010       \\ \hline
  \faAngleDoubleUp      & 011       \\ \hline
  \faAngleDown      & 100       \\ \hline
  \faAngleLeft      & 101       \\ \hline
  \faAngleRight      & 110       \\ \hline
  \faAngleUp      & 111      \\ \hline
\end{tabular}
\caption{Eine andere Zuordnung der Code-Wörter ist auch möglich.}
\label{table-arrows}
\end{table}

\subsection*{Aufgaben 4}

\begin{enumerate}
\item $65\,536$ ergibt $01100101010100110110$. Jede Ziffer wird gemäss der Tabelle codiert. Es braucht pro Ziffer somit $4$ Bits. Total benötigt man für die Zahl $65\,536$ insgesamt $20$ Bits. Mit dem Dualcode würde man die Zahl $65\,536$ als $10000000000000000_2$ darstellen. Man braucht somit $17$ Bits.
\item Man muss die Ziffern in $4$-er Gruppen aufteilen und dann in der Code-Tabelle nachschauen. Man erhält die Zahl $2\,001$.
\end{enumerate}

\subsection*{Aufgaben 5}

\begin{enumerate}
\item $65\,536$ ergibt $00010000000000100000000010000000000010000000100000$. Jede Ziffer wird gemäss der Tabelle codiert. Es braucht pro Ziffer somit $10$ Bits. Total benötigt man für die Zahl $65\,536$ insgesamt $50$ Bits. Mit dem Dualcode würde man die Zahl $65\,536$ als $10000000000000000_2$ darstellen. Man braucht somit $17$ Bits.
\item Man muss die Ziffern in $10$-er Gruppen aufteilen und dann in der Code-Tabelle nachschauen. Man erhält die Zahl $2001$.
\item Bei $10$ Bits gibt es $2^{10}$ Code-Wörter. Es werden \say{nur} 10 Code-Wörter benutzt. Somit bleiben $2^{10} - 10 = 1\,024 - 10 = 1\,014$ Code-Wörter übrig und sind somit nicht erlaubt.
\end{enumerate}

\subsection*{Aufgaben 6}

\begin{enumerate}
\item MENSA ergibt mit der Code-Tabelle 0110000100011011001000000.
\item ABBA
\end{enumerate}

\subsection*{Aufgaben 7}

\begin{enumerate}
\item pro@kswe.ch ergibt 11100001110010110111110000001101011111\\
001111101111100101010111011000111101000
\item 2001: A Space Odyssey
\item Das Code-Wort wird zusammengefügt aus den Bitpositionen. Für das A ergibt sich 1000001. Man kombiniert Zeile und Spalte und fügt die Bits zusammen. Von links nach rechts werden die Bitpositionen wie folgt notiert: 7, 6, 5, 4, 3, 2, 1
\end{enumerate}