\section{\autoref{chapter-dualsystem}, \nameref{chapter-dualsystem}}

\subsection*{Aufgaben 1}

\begin{enumerate}
\item Für grosse Zahlen ist die unäre Zahlendarstellung sehr aufwendig. Man muss zum Beispiel für $65\,536$ genau diese Anzahl an Striche notieren. Dies ist nicht effizient, die Darstellung ist nicht kompakt. 
\item Die Dezimaldarstellung erlaubt es die Zahlen kompakt darzustellen, da wir mehr Ziffern verwenden. Man kann Zahlen kompakter darstellen, wenn man mehr Ziffern verwendet. In Zahlensystemen mit mehr als $10$ Ziffern, wird die Darstellung noch kompakter.
\end{enumerate}


\subsection*{Aufgaben 2}

\begin{enumerate}
\item $67_{10} = 1000011_2$
\begin{alignat*}{6}
67 & : & ~2 & ~=~ & 33 & ~R~ & 1 \\
33 & : & ~2 & ~=~ & 16 & ~R~ & 0 \\
16 & : & ~2 & ~=~ & 8 & ~R~ & 0 \\
8 & : & ~2 & ~=~ & 4 & ~R~ & 0 \\
4 & : & ~2 & ~=~ & 2 & ~R~ & 0 \\
2 & : & ~2 & ~=~ & 1 & ~R~ & 0 \\
1 & : & ~2 & ~=~ & 0 & ~R~ & 1 \\
\end{alignat*}

\item $1000_{10} = 1111101000_2$
\begin{alignat*}{6}
1000 & : & ~2 & ~=~ & 500 & ~R~ & 0 \\
500 & : & ~2 & ~=~ & 250 & ~R~ & 0 \\
250 & : & ~2 & ~=~ & 125 & ~R~ & 0 \\
125 & : & ~2 & ~=~ & 62 & ~R~ & 1 \\
62 & : & ~2 & ~=~ & 31 & ~R~ & 0 \\
31 & : & ~2 & ~=~ & 15 & ~R~ & 1 \\
15 & : & ~2 & ~=~ & 7 & ~R~ & 1 \\
7 & : & ~2 & ~=~ & 3 & ~R~ & 1 \\
3 & : & ~2 & ~=~ & 1 & ~R~ & 1 \\
1 & : & ~2 & ~=~ & 0 & ~R~ & 1 \\
\end{alignat*}

\item $333_{10} = 101001101_2$
\begin{alignat*}{6}
333 & : & ~2 & ~=~ & 166 & ~R~ & 1 \\
166 & : & ~2 & ~=~ & 83 & ~R~ & 0 \\
83 & : & ~2 & ~=~ & 41 & ~R~ & 1 \\
41 & : & ~2 & ~=~ & 20 & ~R~ & 1 \\
20 & : & ~2 & ~=~ & 10 & ~R~ & 0 \\
10 & : & ~2 & ~=~ & 5 & ~R~ & 0 \\
5 & : & ~2 & ~=~ & 2 & ~R~ & 1 \\
2 & : & ~2 & ~=~ & 1 & ~R~ & 0 \\
1 & : & ~2 & ~=~ & 0 & ~R~ & 1 \\
\end{alignat*}

\end{enumerate}

\subsection*{Aufgaben 2}

\begin{enumerate}
\item  $1111_2 = 15_{10}$

\begin{alignat*}{6}
1111_2 &= 1 \cdot 2^0 + 1 \cdot 2^1 + 1 \cdot 2^2 + 1 \cdot 2^3 \\
& = 1 \cdot 1 + 1 \cdot 2 + 1 \cdot 4 + 1 \cdot 8 \\
& = 1 + 2 + 4 + 8 = 15_{10}
\end{alignat*}

\item $100000_2 = 32_{10}$

\begin{alignat*}{6}
100000_2 &= 0 \cdot 2^0 + 0 \cdot 2^1 + 0 \cdot 2^2 + 0 \cdot 2^3 + 0 \cdot 2^4 + 1 \cdot 2^5 \\
& = 0 + 0 + 0 + 0 + 0 + 1 \cdot 32 \\
& = 32_{10}
\end{alignat*}

\item $101101_2 = 45_{10}$

\begin{alignat*}{6}
101101_2 &= 1 \cdot 2^0 + 0 \cdot 2^1 + 1 \cdot 2^2 + 1 \cdot 2^3 + 0 \cdot 2^4 + 1 \cdot 2^5 \\
& = 1 \cdot 1 + 0 + 1 \cdot 4 + 1 \cdot 8 + 0 + 1 \cdot 32 \\
& = 45_{10}
\end{alignat*}

\end{enumerate}

\section*{Aufgaben 3}

\begin{enumerate}

\item Das Bit an der $0$-ten Stelle (ganz \say{rechts}) gibt an, ob eine Zahl gerade oder ungerade ist. Ist das Bit $0$, dann ist die Zahl gerade. Ist das  Bit $1$, dann ist die Zahl ungerade.

\item Eine Dualzahl mit $x$-Stellen entspricht folgender Zahl im Dezimalsystem: $b_0 \cdot 2^0 + b_1 \cdot 2^1 + b_2 \cdot 2^2 + \cdots + b_{x-2} \cdot 2^{x-2} + b_{x-1} \cdot 2^{x-1}$. Mit $b_i$ ist das Bit an der $i$-ten Stelle gemeint. Beispiel: $b_0$ ist das Bit an der $0$-ten Stelle. Wenn wir zum Beispiel die Dualzahl $1010_2$ betrachten, dann erhalten wir folgende Rechnung: $0 \cdot 2^0 + 1 \cdot 2^1 + 0 \cdot 2^2 + 1 \cdot 2^3$. In diesem Beispiel ist $x = 4$, da die Dualzahl vier Bits besitzt.

Wir betrachten eine Dualzahl mit einer $1$ und $n$ Nullen ($1$~$\overbrace{000\dots0}^{n-\textrm{Nullen}}$~$_2$). Die Dualzahl hat somit $x = n + 1$ \textbf{Stellen} (eine Eins und $n$ Nullen). Mit der Rechnung von oben erhalten wir:

\begin{equation*}
b_0 \cdot 2^0 + b_1 \cdot 2^1 + b_2 \cdot 2^2 + \cdots + b_{x-2} \cdot 2^{x-2} + b_{x-1} \cdot 2^{x-1}
\end{equation*}

Mit $x = n + 1$ ergibt dies:

\begin{equation*}
b_0 \cdot 2^0 + b_1 \cdot 2^1 + b_2 \cdot 2^2 + \cdots + b_{n+1-2} \cdot 2^{n+1-2} + b_{n+1-1} \cdot 2^{n+1-1}
\end{equation*}

Da alle Bits ausser das Bit an der $n+1$-ten Stelle (\say{das Bit ganz links}) $0$ sind, können wir die Formel vereinfachen:

\begin{equation*}
0 \cdot 2^0 + 0 \cdot 2^1 + 0 \cdot 2^2 + \cdots + 0 \cdot 2^{n+1-2} + 1 \cdot 2^{n+1-1}
\end{equation*}

Eine Multiplikation mit $0$ ergibt $0$. Wir erhalten: $1 \cdot 2^{n+1-1}$. Wir können den Exponenten vereinfachen: $n+1-1 = n$. Wir erhalten somit folgendes Ergebnis:\\

\textbf{Eine Dualzahl mit einer (führenden) $1$ und $n$ (folgenden) Nullen entspricht der Zahl $2^n$ im Dezimalsystem.}\\

\item Für eine Dualzahl mit $n$ Einsen besitzt die um $1$ grössere Dualzahl eine Eins und $n$ Nullen.\\

Beispiel: Für die Dualzahl $1111_2 (=15_{10})$ ist $10000 (=16_{10})$ die um $1$ grössere Dualzahl.

 Wir können somit eine Dualzahl mit $n$ Einsen mit einer Formel ausdrücken, in dem wir die Dualzahl mit einer Eins und $n$ Nullen berechnen und dann davon $1$ abziehen. Wir verwenden dazu die Formel aus der vorherigen Aufgabe:

\begin{equation*}
\overbrace{111\dots111}^{n-\textrm{Einsen}}~_2 = 1\overbrace{000\dots000}^{n-\textrm{Nullen}}~_2 - 1_2 = (2^n)_{10}-(1)_{10} = (2^n-1)_{10}
\end{equation*}

\textbf{Eine Dualzahl mit $n$ Einsen entspricht der Zahl $2^n-1$ im Dezimalsystem.}

\end{enumerate}