\section{\autoref{chapter-zeichen-und-alphabet}, \nameref{chapter-zeichen-und-alphabet}}

\subsection*{Aufgaben 1}

\begin{enumerate}
\item $\mathscr{A}_{Roemer} = \{I, V, X, L, C, D, M\}$
\item $\mathscr{A}_{Morse} = \{\cdot, -, /\}$
\item $\mathscr{A}_{DNA} = \{A, T, G, C\}$
\item $\mathscr{A}_{Tast-CH} = \{Q, W, @, !, >, \dots\}$
\end{enumerate}

\subsection*{Aufgaben 2}

\begin{enumerate}
\item Es gilt: $\mathscr{A}_{Dez} = \{0, 1, 2, 3, 4, 5, 6, 7, 8, 9\}$. Somit ist ein Wort zum Beispiel $42$. Also kurz: $w_1 = 42$ mit $w_1 \in \mathscr{A}_{Dez}^*$.
\item Nein, da \say{e} und \say{n} nicht in $\mathscr{A}_{Lat} = \{A, B, C, D, \dots , X, Y, Z\}$ vorkommen.
\item 
	\begin{enumerate}
	\item[(a)] $\mathscr{A}_{1}=\{a, b\}$
	\item[(b)] $\mathscr{A}_{2}=\{0, 1, , , (, ), !\}$
	\item[(c)] $\mathscr{A}_{3}=\{a, X, Y, b, u, v, R, S\}$
	\end{enumerate}
\end{enumerate}

\subsection*{Aufgaben 3}

\begin{enumerate}
\item $|w| = |00110011| = 8$
\item Länge 1: U und D; Länge 2: UU, UD, DU, DD; Länge 3: UUU, UUD, UDU, UDD, DUU, DUD, DDU, DDD
\item 
	\begin{enumerate}
	\item[(a)] Die Zahlen aus dem Zahlenschloss bilden das Alphabet. $\mathscr{A}_{Zahlenschloss}=\{0, 1, 2, 3, 4, 5, 6, 7, 8, 9\}=\mathscr{A}_{Dez}$. Eine Kombination für das Zahlenschloss ist ein Wort $w$. Beispiel: $w_1 = 4278$.
	\item[(b)] $10 \cdot 10 \cdot 10 \cdot 10 = 10^4 = 10\,000$ Kombinationen.
	\end{enumerate}
\item 
	\begin{enumerate}
	\item[(a)] Die Buchstaben aus dem Passwort bilden das Alphabet. $\mathscr{A}_{Passwort}=\{a, b, c, \dots , x, y, z\}=\mathscr{A}_{lat}$. Ein konkretes Passwort ist ein Wort $w$. Beispiel: $w_1 = anna$.
	\item[(b)] $26 \cdot 26 \cdot 26 \cdot 26 = 26^4 = 456\,976$ Kombinationen.
	\end{enumerate}
\item Es gilt: $\mathscr{A}_{Bin} = \{0, 1\}$. Somit gibt es $2 \cdot 2 \cdot 2 \cdot 2 = 2^4 = 16$ unterschiedliche Wörter der Länge $4$. Diese lauten: $0000$, $0001$, $0010$, $0011$, $0100$, $0101$, $0110$, $0111$, $1000$, $1001$, $1010$, $1011$, $1100$, $1101$, $1110$, $1111$
\end{enumerate}