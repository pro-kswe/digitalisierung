\toggletrue{image}
\toggletrue{imagehover}
\chapterimage{number_line}
\chapterimagetitle{\uppercase{Number Line}}
\chapterimageurl{https://xkcd.com/899/}
\chapterimagehover{\scriptsize The Wikipedia page List of Numbers opens with \say{This list is incomplete; you can help by expanding it.}}

\chapter{Dezimalsystem}
\label{chapter-dezimalsystem}

Das Dezimalsystem mit seinen zehn Ziffern kennen wir seit dem Schulbeginn. In diesem Kapitel geht es darum, dass wir auf das Zahlensystem und dessen Aufbau eine andere Sichtweise erhalten. Vielleicht haben Sie das Zehnersystem (so wie das Dezimalsystem manchmal auch genannt wird) so noch nie gesehen. Die Lernziele lauten:

\newcommand{\dezimalsystemLernziele}{
\protect\begin{todolist}
\protect\item Sie erklären an einem Beispiel, wie das Zehnersystem aufgebaut ist.
\protect\item Sie \say{zerlegen} eine beliebige Zahl im Zehnersystem in eine Summe von Zehnerpotenzen.
\protect\end{todolist}
}

\lernziel{\autoref{chapter-dezimalsystem}, \nameref{chapter-dezimalsystem}}{\protect\dezimalsystemLernziele}

\dezimalsystemLernziele

\begin{example}

Wir analysieren die Zahl $6314$.

\begin{align*}
\underbrace{631\overbrace{4}^{\textrm{1. Stelle}}}_{\textrm{4 Stellen}} ~ = ~ 6 \cdot 1000 + 3 \cdot 100 + 1 \cdot 10 + 4 \cdot 1
\end{align*}

Die Position einer Ziffer in der Darstellung ist wichtig. Man nennt die Position einer Ziffer \textbf{Stelle}. Jede Stelle hat eine Wertigkeit, den \textbf{Stellenwert}. Die Stelle mit dem niedrigsten Stellenwert steht ganz \textbf{rechts}. Die erste Stelle (ganz rechts) hat den Wert $1$. Die zweite Stelle hat den Wert $10$. Die dritte Stelle hat den Wert $100$ und die vierte Stelle hat den Wert $1000$.
Die Zahlen $123$ und $231$ sind zwei unterschiedliche Zahlen, obwohl Sie die gleichen Ziffern verwenden. Dies liegt daran, dass die Ziffern an unterschiedlichen Stellen notiert sind.

\end{example}

\begin{example}
Wir analysieren die Zahl $2563$.\\

\begin{minipage}{0.6\textwidth}
\begin{alignat*}{16}
2563 & ~ = ~ & 2 & ~\cdot~ & 1000 & ~ + ~ & 5 & ~\cdot~ & 100 & ~ + ~ & 6 & ~\cdot~ & 10 & ~ + ~ & 3 & ~\cdot~ & 1 \\
& ~ = ~ & 2 & ~\cdot~ & 10 \cdot 10 \cdot 10  & ~ + ~ & 5 & ~\cdot~ & 10 \cdot 10  & ~ + ~ & 6 & ~\cdot~ & 10 & ~ + ~ & 3 & ~\cdot~ & 1 \\
& ~ = ~ & 2 & ~\cdot~ & \underbracket{10}_{\Uparrow}{}^3  & ~ + ~ & 5 & ~\cdot~ & \underbracket{10}_{\Uparrow}{}^2  & ~ + ~ & 6 & ~\cdot~ & \underbracket{10}_{\Uparrow}{}^1 & ~ + ~ & 3 & ~\cdot~ & \underbracket{10}_{\Uparrow}{}^0 \\
\omit\rlap{\framebox[9cm]{Basis: $10$ $\Rightarrow$ \textbf{Dezimal}system}}
\end{alignat*}
\end{minipage}
\hfill
\begin{minipage}{0.3\textwidth}
\begin{alignat*}{3}
10^0 & ~ = ~ & 1	\\
10^1 & ~ = ~ & 10	\\
10^2 & ~ = ~ & 100	\\
10^3 & ~ = ~ & 1.000	\\
10^4 & ~ = ~ & 10.000	\\
10^5 & ~ = ~ & 100.000	\\
\end{alignat*}
\end{minipage}
\\
Die Zahl wurde in die ersten vier Zehnerpotenzen aufgeteilt.
\end{example}

\begin{example}

\begin{alignat*}{16}
1050 & ~ = ~ & 1 & ~\cdot~ & 1000 & & & & & ~ + ~ & 5 & ~\cdot~ & 10 \\
& ~ = ~ & \underset{\Uparrow}{1} & ~\cdot~ & 10^3  & ~ + ~ & \underset{\Uparrow}{0} & ~\cdot~ & 10^2  & ~ + ~ & \underset{\Uparrow}{5} & ~\cdot~ & 10^1 & ~ + ~ & \underset{\Uparrow}{0} & ~\cdot~ & 10^0\\
\omit\rlap{\framebox[9cm]{Ziffern des Dezimalsystems: $0, 1, 2, 3, 4, 5, 6, 7, 8, 9$}}
\end{alignat*}

\end{example}

\begin{definition}[Dezimalzahlen]
Die dezimale Darstellung einer Zahl verwendet zehn Zeichen: $0$, $1$, $2$, $3$, $4$, $5$, $6$, $7$, $8$ und $9$. Die Zeichen werden \textbf{Dezimalziffern} genannt. Die Zahlen aus dem Dezimalsystem bestehen aus Dezimalziffern und wir nennen sie \textbf{Dezimalzahlen}.
\end{definition}

\begin{example}

\begin{alignat*}{20}
19605 & ~ = ~ & 1 & ~\cdot~ & 10^{\overset{\Downarrow}{4}}  & ~ + ~ & 9 & ~\cdot~ & 10^{\overset{\Downarrow}{3}}  & ~ + ~ & 6 & ~\cdot~ & 10^{\overset{\Downarrow}{2}} & ~ + ~ & 0 & ~\cdot~ & 10^{\overset{\Downarrow}{1}} & ~ + ~ & 5 & ~\cdot~ & 10^{\overset{\Downarrow}{0}}
\end{alignat*}

Mit jeder Stelle wächst der Exponent um $1$. Bei fünf Stellen gibt es die Exponenten von $0$ bis $4$.

\end{example}

Typischerweise interessiert man sich dafür, wie viele Zahlen man mit einer gewissen Anzahl von Stellen darstellen kann. Im Dezimalsystem kann man dies sehr einfach bestimmen.

\begin{example}

Es sind vier Stellen gegeben. Wie viele (verschiedene) Zahlen kann man darstellen? \\ \\
\textbf{Rechenweg}: $10^4 = 10000$ Zahlen
\begin{itemize}
\item Kleinste Zahl: $0$
\item Grösste Zahl: $10^4-1=9999$
\end{itemize}

\end{example}

Allgemein kann man bei $n$ Stellen folgende \say{Formeln} verwenden:

\begin{itemize}
\item Anzahl Zahlen: $10^n$
\item Kleinste Zahl: $0$
\item Grösste Zahl: $10^n-1$
\end{itemize}